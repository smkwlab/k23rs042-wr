\documentclass[a4paper,12pt]{ujarticle}
\usepackage[top=25mm,bottom=25mm,left=25mm,right=25mm]{geometry}
\title{週間報告}
\author{金武俊佑}
\date{2025年10月08日}

\usepackage{otf}
\usepackage{url}
\usepackage{tabularx}
\usepackage{comment}
\usepackage[dvipdfmx]{graphicx}
\usepackage[dvipdfmx]{color}
\usepackage{listings}
\usepackage{pdfpages}
\usepackage{multirow}
\usepackage{float}

\newcommand{\thc}[1]{\multicolumn{1}{c|}{#1}}

\begin{document}
\maketitle

\section{今週の報告}\label{sec:thisweek}
今週も金曜の1,2限にグループワークの班決めと制作物についての話し合った。\\
 話し合いの結果、生活習慣から身を守るということで、起き上がると同時に電気がつく枕を開発することとなった。マイコンによる電球の操作方法については、先駆者がQiitaで共有している情報を参考にしながら、ハードウェア部分の構築を進めていく予定である。
\begin{figure}[htbp]
  \centering
  \includegraphics[width=10cm]{./img/technoart_1.jpg}
  \caption{試作デザイン}\label{fig:technoart_1}
\end{figure}
\subsection{就職活動}\label{sec:job}
paizaのスカウトに応募する。
\subsection{その他}\label{sec:other}

\section{来週の予定}\label{sec:nextweek}
来週の活動予定を以下に記す。
\begin{itemize}
  \item AtCoderをやる。
  \item テクノアートの回路設計をする。
\end{itemize}


\section{スケジュール}\label{sec:schedule}
後期のスケジュールを表\ref{tab:studyschedule}に示す。

\begin{table}[H]
  \caption{前期のスケジュール}\label{tab:studyschedule}
  \centering
  \begin{tabular}{|p{4mm}||p{30mm}|p{30mm}|p{30mm}|p{30mm}|p{30mm}|}\hline
        & \thc{月} & \thc{火} & \thc{水} & \thc{木} & \thc{金} \\ \hline\hline
     1  &     & AI導入    &情報セキュリティ     & キャリア形成 & テクノアート  \\ \hline
     2  &     & 知的財産権 &  & 現代の政治 & テクノアート \\ \hline
     3  &     & 経営情報学 & 情報科学演習Ⅱ &  &  \\ \hline
     4  &  &  & &  & \\ \hline
  \end{tabular}
\end{table}

\end{document}
