\documentclass[a4paper,12pt]{ujarticle}
\usepackage[top=25mm,bottom=25mm,left=25mm,right=25mm]{geometry}
\title{週間報告}
\author{金武俊佑}
\date{2025年10月28日}

\usepackage{otf}
\usepackage{url}
\usepackage{tabularx}
\usepackage{comment}
\usepackage[dvipdfmx]{graphicx}
\usepackage[dvipdfmx]{color}
\usepackage{listings}
\usepackage{pdfpages}
\usepackage{multirow}
\usepackage{float}

\newcommand{\thc}[1]{\multicolumn{1}{c|}{#1}}

\begin{document}
\maketitle

\section{今週の報告}\label{sec:thisweek}
今週の報告を以下に記す。
\subsection{テクノアート}\label{sec:other}
今週は機械学科の方が用事で来られなかったため、デザインの設計を進めた。特にやることがなかったので芸術学部が設計をやっている所を見学していた。
\subsection{AtCoder}\label{sec:job}
今回はA,B問題が完答できた。
C問題は解説を見ながら時間があるときに解いていきたい。
\subsection{就職活動}\label{sec:job}
弁護士ドットコム株式会社の企業説明会に参加した。その企業のESの提出を求められたのだか、研究内容と開発経験が書けることがないのでどうしようか頭を抱えている。
paizaを使ってほかの企業にも応募していきたい。
\subsection{その他}\label{sec:other}
今週の水曜の一限に中間テストがあるので持ち込み用紙等、念入りに準備をして挑みたい。

下川研のAtCoderのコンテナがいまだに起動できないので空いた時間で原因を調べていきたい。
\section{来週の予定}\label{sec:nextweek}
来週の活動予定を以下に記す。
\begin{itemize}
  \item 11/01(土) 21:00 AtCoderに参加する。
  \item ESを書く
  \item paizaのスカウトに応募する。
\end{itemize}


\section{スケジュール}\label{sec:schedule}
後期のスケジュールを表\ref{tab:studyschedule}に示す。

\begin{table}[H]
  \caption{前期のスケジュール}\label{tab:studyschedule}
  \centering
  \begin{tabular}{|p{4mm}||p{30mm}|p{30mm}|p{30mm}|p{30mm}|p{30mm}|}\hline
        & \thc{月} & \thc{火} & \thc{水} & \thc{木} & \thc{金} \\ \hline\hline
     1  &     & AI導入    &情報セキュリティ     & キャリア形成 & テクノアート  \\ \hline
     2  &     & 知的財産権 &  & 現代の政治 & テクノアート \\ \hline
     3  &     & 経営情報学 & 情報科学演習Ⅱ &  &  \\ \hline
     4  &  &  & &  & \\ \hline
  \end{tabular}
\end{table}

\end{document}
