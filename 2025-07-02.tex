\documentclass[a4paper,12pt]{ujarticle}
\usepackage[top=25mm,bottom=25mm,left=25mm,right=25mm]{geometry}
\title{週間報告}
\author{金武俊佑}
\date{2025年07月02日}

\usepackage{otf}
\usepackage{url}
\usepackage{tabularx}
\usepackage{comment}
\usepackage[dvipdfmx]{graphicx}
\usepackage[dvipdfmx]{color}
\usepackage{listings}
\usepackage{pdfpages}
\usepackage{multirow}
\newcommand{\thc}[1]{\multicolumn{1}{c|}{#1}}

\begin{document}
\maketitle
\section{今週の報告}\label{sec:thisweek}
PBL-L演習を受講するために九州大学の特別聴講生申請の書類を作成した。
\subsection{卒業研究}\label{sec:research}

\subsection{就職活動}\label{sec:job}

\subsection{その他}\label{sec:other}


\section{来週の予定}\label{sec:nextweek}
来週の活動予定を以下に記します。
\begin{itemize}
  \item スケジュール通りに授業を受ける。
  \item データ構造とアルゴリズムⅡの演習問題を解く
  \end{itemize}

\section{スケジュール}\label{sec:schedule}
前期のスケジュールを表\ref{tab:studyschedule}に示す。

\begin{table}[htbp]
  \caption{前期のスケジュール}\label{tab:studyschedule}
  \centering
    \begin{tabular}{|l||l|l|l|l|l|}\hline
        & \thc{月} & \thc{火} & \thc{水} & \thc{木} & \thc{金} \\ \hline\hline
     1  &      &          &        & データベース &   \\ \hline
     2  & オブジェクト指向&データ構造とアルゴリズムⅡ &  &          & 法学\\ \hline
     3  &          &交通システム論 &情報科学演習Ⅰ &          &  \\ \hline
     4  & 生物の世界&人権同和問題&webプログラミング演習&          &心理学の世界 \\ \hline
    \end{tabular}
\end{table}

\begin{comment}
  \begin{thebibliography}{9}
    \bibitem{sanko1} 
    \url{}

  \end{thebibliography}
\end{comment}

\end{document}
