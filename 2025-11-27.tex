\documentclass[a4paper,12pt]{ujarticle}
\usepackage[top=25mm,bottom=25mm,left=25mm,right=25mm]{geometry}
\title{週間報告}
\author{金武俊佑}
\date{2025年11月27日}

\usepackage{otf}
\usepackage{url}
\usepackage{tabularx}
\usepackage{comment}
\usepackage[dvipdfmx]{graphicx}
\usepackage[dvipdfmx]{color}
\usepackage{listings}
\usepackage{pdfpages}
\usepackage{multirow}
\usepackage{float}

\newcommand{\thc}[1]{\multicolumn{1}{c|}{#1}}

\begin{document}
\maketitle

\section{今週の報告}\label{sec:thisweek}
今週の報告を以下に記す。
\subsection{テクノアート}\label{sec:other}
今週は用事があり参加できなかった。フィードバックは共有してもらったので来週改善点について話し合いたい。
\subsection{AtCoder}\label{sec:job}
ぎりぎりまで用事があり途中参加となったため今回は完答はA問題のみとなった。
B,C問題は解説を見ながら時間があるときに解いていきたい。
\subsection{就職活動}\label{sec:job}
今週は用事が多く、あまり余裕がなかったのでスカウトを送れなかった。時間があるときに応募を検討したい。
\subsection{その他}\label{sec:other}
amazonセールで、収納器具を購入した。部屋の整理整頓を進めたい。

\section{来週の予定}\label{sec:nextweek}
来週の活動予定を以下に記す。
\begin{itemize}
  \item 11/29(土) 21:00 AtCoderに参加する。
  \item paizaのスカウトに応募する。
\end{itemize}
\section{スケジュール}\label{sec:schedule}
後期のスケジュールを表\ref{tab:studyschedule}に示す。

\begin{table}[H]
  \caption{後期のスケジュール}\label{tab:studyschedule}
  \centering
  \begin{tabular}{|p{4mm}||p{30mm}|p{30mm}|p{30mm}|p{30mm}|p{30mm}|}\hline
        & \thc{月} & \thc{火} & \thc{水} & \thc{木} & \thc{金} \\ \hline\hline
     1  &     & AI導入    &情報セキュリティ     & キャリア形成 & テクノアート  \\ \hline
     2  &     & 知的財産権 &  & 現代の政治 & テクノアート \\ \hline
     3  &     & 経営情報学 & 情報科学演習Ⅱ &  &  \\ \hline
     4  &  &  & &  & \\ \hline
  \end{tabular}
\end{table}

\end{document}
